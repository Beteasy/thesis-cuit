\section{引言}
    \subsection{课题背景}
        电子资源,就是所有以电子数据的形式把文字、图像、声音、动画等多种形式的信息存贮在光、磁等非纸介质的载体中,并通过网络通信、计算机或终端等方式再现出来的资源。随着计算机应用水平与网络的不断发展,电子资源呈现出传统资源所无法比拟的优势:数量上的海量化、繁多的种类、分布开放、非线性、交互性、共享性、时效性、高增值性等。在发达国家,电子资源建设与传统资源建设的比例为 5:5;在我国部属重点院校电子资源建设与传统资源建设比例为 4:6,而一般院校该比例仅为 2.5 : 7.5。
    \subsection{国内外研究现状}
        从有关的文献看,国外一些图书馆,学术组织和团体机构已经开始研究和探讨电子资源的利用情况和服务效益等问题,也就是电子资源的服务绩效。国内图书馆,由于电子资源数量相对较少,使用时间相对较短,加之服务与成本意识较为薄弱,电子资源建设的质量与服务绩效等问题尚未引起人们的重视。国内注重电子资源的质量评价,而国外注重电子图书馆的服务绩效评估 \upcite{索传军2005电子资源服务绩效评估的含义及影响因素分析}。
    \subsection{本课题的研究方法}
        建立读者满意度指标体系,通过对指标体系的层层展开形成调查问卷,借助5级李克特量表的方法对问卷上的问题进行量化评分。采用分层抽样和随机抽样相结合的方法,对学院的150名师生进行问卷调查和数据收集,并使用EXCEL软件对数据进行统计和频率分布等分析。结合统计好的数据与事先建立的测评指标体系计算读者满意度值,进行评价分析,并提出相应的改进建议及措施。